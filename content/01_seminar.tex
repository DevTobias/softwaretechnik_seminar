%\begin{fullpage}
	
\section{Seminar - Werkzeuge Softwareentwicklung}

    \Aufgabe{1 - Werkzeugkategorien}
	
	   \begin{aufgabe}
           In der Softwaretechnik gibt es eine große Menge an Programmen und Systemen, die bei der Entwicklung eines Softwareprojekts verwendet werden.
           \\[-.7cm]\begin{itemize}
               \setlength\itemsep{0.1px}
               \item In welche Kategorien können die Werkzeuge unterteilt werden? Nennen Sie jeweils
               ein Beispiel.
               \item Welchen Zweck verfolgen Werkzeuge der einzelnen Kategorien?
           \end{itemize}
       \end{aufgabe}
   
        \begin{loesung}\:
            \\[-.7cm]\begin{enumerate}
                \setlength\itemsep{0.1px}
                \item \textcolor{Orange}{Modellierungs-Tools} (UML) - Abstrakte Möglichkeit Software und Systeme (Architektur, Systemdesign) zu beschreiben ohne irrelevante Details zu beachten. Solche Abstraktionen haben folgenden Zweck: Dokumentation, Kommunikation, Spezifikation, Vertragsabschluss und Ausschreibung.
                \item \textcolor{Orange}{Implementierungs-Tools} (Java, OOP, MongoDB) - Ermöglichen dem Programmierer mit Abstraktionen der Formen Programmiersprachen, Programmierparadigmen, Datenbanken und vieles mehr, das Modell technisch einfach zu realisieren über die Schnittstelle des Tools. Dabei können die Programmiersprachen erneut abstrahiert werden mittels Frameworks/Stacks, damit häufige Probleme nicht immer wieder erneut implementiert werden müssen (ReactJS).
                \item \textcolor{Orange}{Entwicklungsplattformen/Editoren} (IntelliJ IDEA, Visual Studio Code) - Einfache Umgebungen (Texteditoren) helfen den Programmierer, Quelltext einer konkreten Programmiersprache zu formulieren oder Änderungen vorzunehmen. Auch spezielle GUI-Editoren sind dabei denkbar. Integrierte Entwicklungsumgebungen sind dabei viel komplexer und übernehmen mehrere Aufgaben dieser Liste gleichzeitig.
                \item \textcolor{Orange}{Build-Tools} (Compiler) - Die konkret Implementierten Systeme müssen kompiliert oder interpretiert werden, damit ein Computersystem diese auch ausführen können
                \item \textcolor{Orange}{Versionsverwaltung} (Git) - Speicherung und Verwaltung von Programmen mit mehreren Versionen um viele Backups zu erstellen und den Entwicklungsprozess zu dokumentieren.
                \item \textcolor{Orange}{Programmverwaltung} (JUnit, Githooks) - In Programmen können Fehler auftauchen, die nicht einfach zu entdecken sind, weswegen Debugging-Tools benötigt werden. Um solche Fehler rechtzeitig zu entdecken, müssen die Programme getestet werden, damit sichergestellt werden kann, dass das Programm das erwartete Verhalten aufweist. Diese Tests können automatisch vor jedem Hinzufügen zur Versionsverwalten gesehenen, damit kein falscher Quelltext gespeichert wird (Deployment).
            \end{enumerate}
        \end{loesung}
	
    
	\Aufgabe{2 - Github}
    
        \begin{aufgabe}
            Machen Sie sich mit den Möglichkeiten zur Planung von Softwareprojekten am Beispiel von Github vertraut.
            \\[-.7cm]\begin{enumerate}[(a)]
                \setlength\itemsep{0.1px}
                \item  Welche Instrumente der Projektplanung bietet die Plattform Github und was sind
                deren wichtigste Funktionen und Eigenschaften?
                \item Welche Personengruppen können am Prozess der Projektplanung beteiligt sein?

            \end{enumerate}
        \end{aufgabe}
    
        \begin{loesung}\:
            
        \end{loesung}
	
    
    \Aufgabe{3 - Versionsverwaltung}
    
        \begin{aufgabe}
            Versionierung ist ein zentrales Element bei der konsistenten Gruppenarbeit an Projekten.
            \\[-.7cm]\begin{enumerate}[(a)]
                \setlength\itemsep{0.1px}
                \item Welche Funktionen bietet eine Versionsverwaltung und welche Vorteile ergeben sich für die Entwicklung?
                \item Was bedeuten die folgenden Begriffe im Kontext der Versionsverwaltung der GitSoftware?
                \\[-.7cm]\begin{itemize}
                    \setlength\itemsep{0.1px}
                    \item Branch, Checkout, Pull-Request \& Merge, Tags
                \end{itemize}
                \item Wie sollte die Versionierung während der Entwicklung verwendet werden?
            \end{enumerate}
        \end{aufgabe}
    
        \begin{loesung}\:
            
        \end{loesung}
	
    
	\Aufgabe{4 - Entwicklungsumgebungen}
    
        \begin{aufgabe}
            Welche Funktionen bringt eine Entwicklungsumgebung (IDE) mit und wie integrieren sie sich in den Entwicklungsprozess?
        \end{aufgabe}
    
        \begin{loesung}\:
            
        \end{loesung}
	
    
	\Aufgabe{5 - Dokumentation}
    
        \begin{aufgabe}
            Dokumentation ist ein Teil des kompletten Entwicklungsprozesses.
            \\[-.7cm]\begin{enumerate}[(a)]
                \setlength\itemsep{0.1px}
                \item Nennen Sie verschiedene Arten der Dokumentation. Welche Programme oder Hilfsmittel können dabei verwendet?
                \item Welche Rolle spielen Modellierung und UML in diesem Prozess?
            \end{enumerate}
        \end{aufgabe}
    
        \begin{loesung}\:
            
        \end{loesung}
	
%\end{fullpage}